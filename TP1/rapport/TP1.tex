\documentclass{article}
\usepackage[pdftex]{graphicx}
\usepackage{hyperref}
\usepackage{amsmath}

\usepackage[utf8x]{inputenc}
\usepackage[T1]{fontenc}
\usepackage[frenchb]{babel}
\usepackage{float}
\usepackage{listings}

% for tables
\usepackage{booktabs}
\usepackage{multirow}

% dendogram
\usepackage{tikz}


\begin{document}

\title{Introduction à la vision par ordinateur \\
TP1: Détection de visages}
\author{Romain Catajar - romain.catajar@student.ecp.fr}
\maketitle

\begin{abstract}
Présentation d'implémentations de différentes méthodes de détection de peau ou de visage et analyse des résultats. Le code a été réalisé avec l'aide de deux autres étudiants: Léo Hoummady et Thibault Kaspi.
\end{abstract}
% skip section 1 to 3
\stepcounter{section}
\stepcounter{section}
\stepcounter{section}
\section{Approche 1: Detection de la peau}
\subsection{Avant de commencer}
\begin{enumerate}
    \item Que pensez vous de ce type d'approches, que l'on pourra qualifier
d'approches colorimétriques, pour la détection de visages ? La détection
de la peau est-elle suffisante ? Quelles en sont les limites ?\\\\
Cette approche me semble être limité. En effet, même si on imagine pouvoir détecter parfaitement les pixels de peau, cela ne suffit pas a détecter le visage (le visage n'est pas forcement la seule peau apparaissant sur une image). De plus, je ne suis pas convaincu qu'une approche colorimetrique est suffisante pour detecter efficacement les pixels de peau. En effet la couleur des pixels de peau peux varier grandement d'une image a une autre en fonction de plusieurs critères (exposition et luminosité de la photo, ethnicité de la personne sur la photo, ...). Il me semble qu'une approche efficace devrait prendre en compte le contexte autour des pixels pour dire s'il s'agit de peau ou non (par exemple, en regardant le gradient de l'image, la couleur des pixels de peau devant a priori varier peu).\\

    \item Dans la mise en oeuvre de ce type d'approches, quelles sont les principales questions à se poser ? \\
    \begin{itemize}
        \item Qualité et variété des images utilisées (différents éclairages et couleurs de peau par exemple)
        \item Choix du modèle de séparation entre peau et non peau
        \item Choix de l'espace de couleur
    \end{itemize}
\end{enumerate}

\subsection{A vous de jouer}
\subsubsection{Base d'images}
J'ai construit une base de 78 images a partir du dataset \emph{Pratheepan Dataset}. Ces images sont ensuite séparées en deux groupe, un groupe "d'entrainement" de 52 images et un groupe de test de 26 images.

\subsubsection{Une première méthode simple}
Les résultats obtenues sont présenté dans le tableau ci dessous. Globalement le taux de bonne détection est acceptable, mais celui de mauvaise détection est trop élevée. Peut être que cette règle est trop permissive.

TODO: Ajouter tableau, Ajouter exemples

\subsubsection{Approche non paramétrique: Histogramme et modèle de peau}
J'ai construit les histogramme a partir du dataset d'entrainement pour les espaces RGB, LAB et HSV. 

TODO: Ajouter tableau, Ajouter exemples

\subsubsection{Méthode de Bayes}
J'utilise les histogrammes de la partie précédente pour mon modèle de Bayes. Pour déterminer le meilleur seuil, je cherche a maximiser le taux de bonne detection tout en minimisant celui de mauvaise detection. Pour cela, j'ai fait le choix de tester différentes valeurs de seuil en cherchant a maximiser leur différence. J'obtient les valeurs de seuil suivante:
\begin{itemize}
    \item \textit{RGB}: $0.2$
    \item \textit{LAB}: $0.15$
    \item \textit{HSV}: $0.15$
\end{itemize}
Des seuils plus faibles permettent d'augmenter le taux de bonne détection, mais rapidement le taux de mauvaise détection grimpe au dessus des 20\%

TODO: Tableau résultats, exemples

\subsubsection{Évaluation}

TODO: recap, comparaison histo et bayes. 

\subsubsection{Discussion}

TODO: proposer une ou deux ameliorations

\section{Detection de visages par l'approche de Viola Jones}
\subsection{Etude de la methode}
TODO: Code, exemples, resultats

\subsection{Discussion: pour aller plus loin!}
TODO: pipo (methodes de deep learning)

\section{Detection de visages par des algorithmes de segmentation}
\end{document}
